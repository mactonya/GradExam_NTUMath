
\begin{center}
    \textbf{National Taiwan University Dept. of Mathematics \\ Graduate Entrance Exam 2013: Advanced Calculus/Analysis \\ Solutions} 
    \noindent\rule{\textwidth}{0.4pt}
\end{center}
\begin{enumerate}
    \item The integral can be rewrite as
    \[
    \int_{-\infty}^\infty \int_{-\infty}^\infty \frac{e^{-x^2}}{1+(x+y)^2}dxdy
    \]
    by noting that the denominator does not change along the line $x+y=c$ for $c \in \mathbb{R}$, we can take a change of variable $z = x+y$, so that
    \begin{align*}
        \int_{-\infty}^\infty \int_{-\infty}^\infty \frac{e^{-x^2}}{1+(x+y)^2}dxdy &= \int_{-\infty}^\infty \int_{-\infty}^\infty \frac{e^{-x^2}}{1+z^2}dxdz \\ 
        &= \int_{-\infty}^\infty \int_{-\infty}^\infty \frac{e^{-x^2}}{1+z^2}dzdx \\
        &= \int_{-\infty}^\infty e^{-x^2} dx \tan^{-1}z\vert^\infty_{z = -\infty} \\ 
        &= \pi \int_{-\infty}^\infty e^{-x^2} dx  \\ 
        &= \pi \cdot \sqrt{\pi} =  \pi^{\frac{3}{2}}
    \end{align*}
    The swap of double integral is guaranteed by Tonelli's theorem. $\qed$
    \item We abbreviate $g(x,y)$ as $g$ and $\bar{g}(x)$ as $\bar{g}$ for simplicity.
    \begin{enumerate}
        \item Notice that $\log\left(\frac{1+g}{1-g}\right)$ and $\tan^{-1}(yg)$ are increasing (and decreasing) simultaneously with $g$. As $x$ increases, the left hand side of the equation must increase, and $g$ must increase.
        \item Rewrite the equation as
        \[
        \log\left(\frac{1+g}{1-g}\right) = 2(y^2+1)x - 2y \tan^{-1}(yg)
        \]
        and consider 3 separate cases:
        \begin{enumerate}
            \item If $x > 0$, then the right hand side tend to $+\infty$ (as $y^2$ tends to $\infty$ faster than $y$, and $\tan^{-1}$ is bounded), therefore by letting $y \to \infty$ in this equation we have
            \[
            \log\left(\frac{1+\bar{g}}{1-\bar{g}}\right) \to +\infty
            \]
            and solving $\bar{g}$ gives $\bar{g} = 1$.
            \item If $x < 0$, then the right hand side tend to $-\infty$, therefore by letting $y \to \infty$ in this equation we have
            \[
            \log\left(\frac{1+\bar{g}}{1-\bar{g}}\right) \to -\infty
            \]
            and solving $\bar{g}$ gives $\bar{g} = -1$.
            \item If $x = 0$, then we have
            \begin{align*}
                \log\left(\frac{1+g}{1-g}\right) &= - 2y \tan^{-1}(yg) \\
                \frac{1+g}{1-g} &= e^{- 2y \tan^{-1}(yg)} \\
                1+g &= (1-g) e^{- 2y \tan^{-1}(yg)} \\
                g &= \frac{e^{- 2y \tan^{-1}(yg)} - 1}{1 + e^{- 2y \tan^{-1}(yg)}}
            \end{align*}
            and we have $\bar{g}(0) = 0$, which could be verified easily by calculation. \\
            (If $\bar{g}(0) > 0$, then the right hand side tends to $-1$, which is absurd; if $\bar{g}(0) < 0$, then the right hand side tends to $1$ ($\tan^{-1}$ changes sign), which is again, nonsense) 
        \end{enumerate}
        Therefore
        \[\bar{g}(x) = \begin{cases}
        1 &\text{\quad if } x > 0 \\
        0 &\text{\quad if } x = 0 \\
        -1 &\text{\quad if } x < 0 \\
        \end{cases}.\]
        \item To show that $g$ is differentiable, we consider
        \[
        F(x,y,g(x,y)) = \log\left(\frac{1+g(x,y)}{1-g(x,y)}\right) + 2y \tan^{-1}(yg(x,y)) - 2(y^2+1)x
        \]
        and
        \[
        \frac{\partial F}{\partial g} = \left( \frac{2}{(1-g)^2} \right) \frac{1-g}{1+g} + 2y^2 \frac{1}{1+y^2g^2} = \frac{2}{1-g^2} + \frac{2y^2}{1+y^2g^2} 
        \]
        which is nowhere zero within the range. It is easily seen that we can always find $g(x,y)$ such that $F(x,y,g(x,y)) = 0$ for all $(x,y)$ (as $\log(\frac{1+g}{1-g})$ maps $(-1,1)$ to $\mathbb{R}$), therefore by implicit function theorem, $g$ is differentiable in $(x,y)$.
        \item Take $\frac{\partial}{\partial x}$ on both sides of equation gives $(g' = \frac{\partial}{\partial x}g)$
        \[
        \left( \frac{2g'}{(1-g)^2} \right) \frac{1-g}{1+g} + 2y^2g' \frac{1}{1+y^2g^2} = 2(y^2+1)
        \]
        \[
        g' = \left(1-g^2\right)\left(1+y^2g^2\right)
        \]
        By taking limit, we have
        \[
        \lim_{y\to \infty} g' = \lim_{y \to \infty} (1-\bar{g}^2)(1+y^2\bar{g}^2)
        \]
        and after some calculation we conclude that
        \[
        \lim_{y\to \infty} \frac{\partial g}{\partial x}(x) = 
        \begin{cases}
        1 &\text{\quad if } x \neq 0, \\
        0 &\text{\quad if } x = 0. \\
        \end{cases}
        \]
        The limit can be calculated using L'Hôpital's rule. $\qed$
    \end{enumerate}
    \item 
    \begin{enumerate}
        \item If we write
        \[
        f(x) = \sum_{n \in \mathbb{Z}} a_n e^{inx}
        \]
        then the fourier coefficient is
        \[
        a_n = \frac{1}{2\pi} \int_{-\pi}^\pi f(x) e^{-inx} dx = \frac{1}{2\pi} \; 2 \int_0^\pi e^{-in\pi} dx = \frac{-1}{in\pi} e^{-inx}\vert^\pi_{x = 0} = 
        \begin{cases}
        0 &\text{ if \emph{n} is even}  \\
        \frac{2}{inx} &\text{ if \emph{n} is odd}
        \end{cases}
        \]
        therefore
        \[
        f(x) = \frac{2}{i\pi} \sum_{n \in \mathbb{Z}} \frac{1}{(2n+1)} e^{i(2n+1)x}.
        \]
        \item Let $x = \frac{\pi}{2}$ and we have
        \[
        1 = \frac{2}{i\pi} \left(\cdots \frac{1}{-5}(-i) + \frac{1}{-3}i + \frac{1}{-1}(-i) + \frac{1}{1}i + \frac{1}{3}(-i) + \frac{1}{5}i + \cdots\right)
        \]
        therefore
        \[
        1 = \frac{\textbf{4}}{\pi}\left( 1 - \frac{1}{3} + \frac{1}{5} - \frac{1}{7} + \cdots \right)
        \]
        and the desired sum is $\dfrac{\pi}{4}. \qed$
    \end{enumerate}
    \item
    \begin{enumerate}
        \item Proceed by mathematical induction, the case $n = 2$ is simple, so we assume that $f_k < f_{k-1}$ for some $k$. This gives
        \[
        f_{k}(x) = \dfrac{1}{2}\left(f_{k-1}(x) + \dfrac{e^x}{f_{k-1}(x)}\right) < f_{k-1}(x)
        \]
        \begin{equation*}
        f_{k-1}(x)^2 - e^x > 0 \tag{1}
        \end{equation*}
        Now we want to show that $f_k(x)^2 - e^x > 0$, so $f_{k+1} < f_{k}$. Indeed,
        \begin{align*}
            f_k(x)^2 - e^x &= \frac{1}{4}\left(f_{k-1}(x) + 2e^x + \frac{e^{2x}}{f_{k-1}(x)}\right) - e^x \\
            &= \frac{1}{4}\left(f_{k-1}(x)^2 - 2e^x + \frac{e^{2x}}{f_{k-1}(x)^2}\right) \\
            &\geq \frac{1}{4}\left(- e^x + \frac{e^{2x}}{f_{k-1}(x)^2}\right) \quad \quad \text{by (1)} \\
            &\geq \frac{1}{4}e^x\left(1 - \frac{e^x}{f_{k-1}(x)^2}\right) > 0
        \end{align*}
        as $\dfrac{e^x}{f_{k-1}(x)^2} < 1$ by (1).
        \item Take limit on both sides of the recursive formula leaves
        \[
        f(x) = \frac{1}{2} \left( f(x) + \frac{e^x}{f(x)} \right)
        \]
        and solving this gives $f(x) = e^{\frac{x}{2}}$.
        \item Yes, as $[-1,1]$ is compact, $f_n(x)$ is continuous for all $n$, and $\{f_n\}$ is monotone decreasing in $n$, therefore by Dini's Theorem, the sequence converges to $f$ uniformly. $\qed$
    \end{enumerate}
\end{enumerate}
