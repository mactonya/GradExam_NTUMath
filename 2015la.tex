
\begin{center}
    \textbf{National Taiwan University Dept. of Mathematics \\ Graduate Entrance Exam 2015: Linear Algebra \\ Solutions} 
\noindent\rule{\textwidth}{0.4pt}
\end{center}
\begin{enumerate}
    \item 
    If $v$ is a vector that is in $V \cap W$, then this means that there are values $a, b\in \mathbb{R}^4$ such that
    \[
    v = \begin{pmatrix}1&2&3&4&5&6\\3&4&6&7&9&10\\0&1&0&2&0&3\\1&-2&3&-4&5&-6\end{pmatrix}^T a = 
    \begin{pmatrix}1&1&1&2&2&3\\-2&0&-1&0&1&2\\1&0&1&0&2&0\\0&0&1&0&-2&-2\end{pmatrix}^T b
    \]
    and, in a complete matrix, 
    \[
    \begin{pmatrix}1&3&0&1&1&-2&1&0\\2&4&1&-2&1&0&0&0\\3&6&0&3&1&-1&1&1\\4&7&2&-4&2&0&0&0\\5&9&0&5&2&1&2&-2\\6&10&3&-6&3&2&0&-2\end{pmatrix}
    \begin{pmatrix}a_1\\a_2\\a_3\\a_4\\-b_1\\-b_2\\-b_3\\-b_4
    \end{pmatrix} = 0
    \]
    \item We show that \begin{center}
        given $A$ a integer matrix, $A^{-1}$ is an integer matrix if (and only if) $\det (A) = \pm 1$	
    \end{center}
    Indeed, if $\det(A) = \pm1$, then $\det(A^{-1}) = \mp 1$. Since all the adjugate matrix has integer entries, $A^{-1}$ is a integer matrix. So we can calculate the determinant of $A$, which is 
    \[	
    \det A = 2x^2 - x - 4
    \]
    as $\det A$ can only be $-1$ (the other one has no integer solution), we have $\det A = -1$, which has a integer solution $x= - 1$. Then the inverse matrix is found as 
    \[
    \begin{pmatrix}
    0&-4&2&-9&-5\\1&-1&2&-6&-1\\0&3&-2&8&4\\0&0&0&-1&0\\0&-2&1&-4&-3
    \end{pmatrix}. \qed
    \]
    \item A linear operator $T:V \to V$ is diagonalizable if and only if all the vectors can be written as linear combinations of eigenvectors.
    A property that the transpose operator has is that $T^2 = I$. In other words, $T$ has eigenvalue $+1$ and $-1$.
    Now, if $M$ is any matrix, then $M + M^t$ is a eigenvector of $+1$, and $M - M^t$ is a eigenvector of $-1$. As we have
    \[
    M = \frac{1}{2}(M + M^t) + \frac{1}{2}(M - M^t)	
    \]
    we conclude that $T$ is diagonalizable. Notice that if $\text{char } F = 2$, then the only eigenvalue is $1$, and since the dimesion of all symmetric matrices are only $\frac{n(n+1)}{2}$, the multiplicities did not agree, hence not diagonalizable. \\
    To find a basis, simply list out a basis for symmetric matrices and a basis for antisymmetric matrices and take the union. $\qed$
    \item If $T$ is onto, then we want to check if $\ker T^* = \{0\}$ or not. This follows since if $f(T) = 0$, then
    \[
    f(w) = f(T(v)) = 0 \; \forall w \in W 
    \]
    by surjectivity of $T$. So $f$ must be $0$. 

    For the converse, if we have $T^*$ injective, we need to show $\im T = W$. Suppose not, i.e. $\im T \subsetneq W$, then we can construct a linear functional $f \in W^*$ by
    \[
    blablabla
    \]
    \item Consider 
    \[0 = \det (cA+B) = \det (cI + BA^{-1}) \det(A)\]
    This yields $\det (cI + BA^{-1}) = 0$ as $A$ is invertible. This equation then just simply calculate the characterstic polynomial of $-BA^{-1}$. As the degree of the polynomial is at most $n$, there are at most $n$ number such that the matrix $cA+B$ is not invertible. $\qed$
    \item
    \begin{enumerate}
        \item 
        \[
        q(TS) = TSp(TS) = Tp(ST)S = 0.	
        \]
        \item Either the minimal polynomial are equal ($ST = TS$), or they differ from a multiple.
    \end{enumerate}
    \item Existence of the vector follows by consider the vector
    \[
    v = \sum_1^n c_n v_n.	
    \]
    Now if there are two vectors $v, v'$ that satisfy the equations given, then
    \[
    0 = \langle v, v_j \rangle - \langle v', v_j \rangle = \langle v-v', v_j \rangle 	
    \]
    for all $j = 1, \cdots n$. This simply says that $v - v'$ is orthogonal to all vectors that form a basis, which implies $v - v' = 0. \qed$
\end{enumerate}