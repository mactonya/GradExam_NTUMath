
\begin{center}
    \textbf{National Taiwan University Dept. of Mathematics \\ Graduate Entrance Exam 2014: Linear Algebra \\ Solutions} 
    \noindent\rule{\textwidth}{0.4pt}
\end{center}

\begin{enumerate}
    \item The first problem is flawed as the solution space of the given two equations is only 1-dimensional.
    \item 
    \begin{enumerate}[label=(\arabic*)]
        \item $f(\lambda) = \det(A - \lambda I) = - \lambda^{3} + 7 \lambda^{2} - 15 \lambda + 9$. 
        \item The solutions of $f$ are $1$ and $3$, in which $3$ has multiplicity of 2. If $\lambda = 1$ we have
        \[ A-I = 
        \begin{pmatrix}
        0 & 2 & 2 \\
        1 & 1 & -1 \\
        -1 & 1 & 3
        \end{pmatrix}
        \]
        and a possible eigenvector is 
        \[
        e_1 = 
        \begin{pmatrix}
        -2 \\ 1 \\ -1
        \end{pmatrix}
        .\]
        When $\lambda = 3$ we have
        \[ A-3I = 
        \begin{pmatrix}
        -2 & 2 & 2 \\
        1 & -1 & -1 \\
        -1 & 1 & 1
        \end{pmatrix}
        \]
        which has two eigenvectors as the rank is $1$:
        \[
        e_2 = 
        \begin{pmatrix}
        1 \\ 0 \\ 1
        \end{pmatrix}, \quad
        e_3 = 
        \begin{pmatrix}
        1 \\ 1 \\ 0
        \end{pmatrix}.
        \]
        Therefore we have
        \[
        P = 
        \begin{pmatrix}
        -2 & 1 & 1 \\
        1 & 0 & 1 \\
        -1 & 1 & 0
        \end{pmatrix}
        . \qed\]
    \end{enumerate}
    \item
    \begin{enumerate}[label=(\arabic*)]
        \item If $A^k = 0$, then $I - A^k = I$, and 
        \[
        (I-A)(I+A+A^2+A^3+\cdots+A^{k-1}) = I - A^k = I
        \]
        so $I-A$ is invertible.
        \item First note that $\im(A^{k}) \subseteq \im(A^{k-1})$ since $A^k(V) = A^{k-1}(\im A) \subseteq A^{k-1}(V)$, so $\rank (A^{k}) \leq \rank(A^{k-1})$. But as $V$ is finite dimensional, the sequence of descending subsets (or ranks) must stop, meaning there exists some $p$ such that
        \[
        \rank (A^{p}) = \rank(A^{p+k}) \quad \text{ i.e. } A^p(V) = A^{p+k}(V) \quad \forall k \geq 0 \tag{1}
        \]
        Now consider the simplest nontrivial case, i.e. $p = 2$. If $\rank(A) = \rank(A^2)$, then 
        \[
        \dim(\im A) = \rank A = \rank A^2 = \dim(A(A(V)) = \dim(A(\im A)) = \rank( A\vert_{\im A})
        \]
        By viewing $A\vert_{\im A}$ as a endomorphism on $\im A$ (as $\im A$ is $A$-invariant), since the rank of $A$ is exactly $\im A$, we conclude that $A\vert_{\im A}$ is surjective (hence injective since $V$ is finite dimensional), so $\{0\} = \ker A\vert_{\im A} = \ker A \cap \im A$. Then it is immediate that
        \[
        V = \ker A \oplus \im A.
        \]
        Then for any $p \geq 2$ that satisfy (1), since $\rank (A^{p}) = \rank(A^{2p})$, by the simple case we have
        \[
        V = \ker A^p \oplus \im A^p.
        \]
        as desired. $\qed$
    \end{enumerate}
    \item
    Just assume that $X$ is of form $\begin{pmatrix} a & b \\ c & d \end{pmatrix}$, and after calculation we have
    \[
    L\begin{pmatrix} a & b \\ c & d \end{pmatrix} = \begin{pmatrix} -c-4b & 5b-d+a \\ 4a-5c-4d & 4b+c \end{pmatrix}
    \]
    and if we identify the $2 \times 2$ matrix by $\mathbb{R}^4$ then we have
    \[
    L\begin{pmatrix} a \\ b \\ c \\ d \end{pmatrix} = 
    \begin{pmatrix} 
    0&-4&-1&0\\1&5&0&-1\\4&0&-5&-4\\0&4&1&0
    \end{pmatrix}
    \begin{pmatrix} a \\ b \\ c \\ d \end{pmatrix}
    \]
    To find a basis of $\im L$ we can plug in the standard $\mathbb{R}^4$ basis and simplify, and we have
    \[
    \left\{ \begin{pmatrix} 0\\1\\4\\0 \end{pmatrix}, \begin{pmatrix} -1\\0\\5\\1 \end{pmatrix} \right\}
    \]
    as a basis of $\im L$. Then by Dimension Theorem, the dimension of the kernel is $2$. $\qed$
    \item
    \begin{enumerate}[label=(\arabic*)]
        \item Let $S = AA^*$ so $S = S^*$. Then $AA^*v = A^*Av = 0 = Sv$. Then 
        \[
        \langle Av, Av \rangle = \langle v, Sv \rangle = 0
        \]
        therefore $Av = 0$.
        \item Let $k$ be the minimal positive integer such that $A^kv = 0$. Then
        \[
        S^kv = (A^*A)^kv = A^{*k}A^kv = 0
        \]
        so
        \[
        0 = \langle S^kv, S^{k-2}v \rangle = \langle S^{k-1}v, S^{k-1}v \rangle
        \]
        and $S^{k-1}v = 0$. Then following this pattern we have $Sv = 0$, and by the previous question $Av = 0$.
        \item If the minimal polynomial of $A$, say $f$, has repeated roots, i.e. $O = f(A) = (A-kI)^p g(A)$ where $p>1$, $g(A)$ is some polynomial, then for all  $v \in \im g(A)$, we have
        \[
        0 = (A-kI)^p v
        \]
        Then previous problem and the fact that $(A-kI)$ is normal for all $k$ impiles that $(A-kI)v = 0$ for all $v \in \im g(A)$, therefore
        \[
        (A - kI)g(A) = O
        \]
        but this polynomial has degree less than $f$, contradicting to the hypotheses that $f$ is minimal. Hence the charactristic polynomial of $A$ must not have repeated roots. $\qed$
    \end{enumerate}
\end{enumerate}