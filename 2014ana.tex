
\begin{center}
    \textbf{National Taiwan University Dept. of Mathematics \\ Graduate Entrance Exam 2014: Advanced Calculus/Analysis \\ Solutions} 
    \noindent\rule{\textwidth}{0.4pt}
    \end{center}
    
    \begin{enumerate}
        \item The function
        \[
        f(x) =
        \begin{cases}
        x^2\cos(\frac{1}{x}), &x \neq 0 \\
        0, &x = 0
        \end{cases}
        \]
        is differentiable and has derivative
        \[
        f'(x) =
        \begin{cases}
        2x\cos(\frac{1}{x}) + \sin (\frac{1}{x}), &x \neq 0 \\
        0, &x = 0
        \end{cases}
        \]
        but $\lim_{x \to 0} f'(x)$ does not exist as sin oscillates near 0, therefore not continuous. $\qed$
        \item For every $\epsilon > 0$ we must find $\delta > 0$ such that
        \[
        \left\vert \frac{x}{\sqrt{x^2+5}} - \frac{2}{3} \right\vert < \epsilon \text{ whenever } |x - 2| < \delta	\tag{*}
        \]
        Consider the "close" case $|x - 2| < 1$, i.e. $1 < x < 3$. Then
        \begin{align*}
        \left\vert \frac{x}{\sqrt{x^2+5}} - \frac{2}{3} \right\vert &= \left\vert \frac{3x - 2\sqrt{x^2+5}}{3\sqrt{x^2+5}} \right\vert \\
        &= \left\vert \frac{(3x - 2\sqrt{x^2+5})(3x + 2\sqrt{x^2+5})}{(3\sqrt{x^2+5})(3x + 2\sqrt{x^2+5})} \right\vert \\
        &= \left\vert \frac{5x^2-20}{(3\sqrt{x^2+5})(3x + 2\sqrt{x^2+5})} \right\vert
        \end{align*}
        Now notice that the numerator is 
        \[
        |5x^2 - 20| = 5|(x+2)(x-2)| < 25 |x-2|
        \]
        and the denominator is
        \[
        |(3\sqrt{x^2+5})(3x + 2\sqrt{x^2+5})| > 9\sqrt{5} + 30 > 50
        \]
        for all $x \in (1,3)$. Now we can choose
        \[
        \delta = \min (1, \epsilon /2)	
        \]
        and it will satisfy $(^*)$.  $\qed$
        \item False. We can construct a counterexample
        \[
        n_1 = (1,0,0,\cdots), n_2 = (0,1,0,0, \cdots), n_3 = (0,0,1,0,0,\cdots), \cdots
        \]
        The sequence $\{n_i\}_{i=1}^\infty$ is in $B$, but $B$ is not (sequentially) compact as $\{n_i\}_{i=1}\infty$ does not have any convergent subsequence.  $\qed$
        \item False. Only if the metric space is \textit{complete}, then any Cauchy sequence will converge in $X$. A counterexample is given by an non-complete metric space
        \[
        X = (\mathbb{Q}, d) \quad \text{ where } \quad d(x,y) = |x-y|
        \]
        then there are seqences in $X$ that converges to $\sqrt{2}$, but $\sqrt{2} \notin \mathbb{Q}$. $\qed$
        \item False. Choose a partition $0 = a_0 < a_1 < \cdots < a_n = 1$. Then we can always find $k$ even such that
        \[
        0 < \frac{1}{(k+\frac{3}{2})\pi} < \frac{1}{(k+\frac{1}{2})\pi} < a_1
        \]
        by Archimedean principle. Then $f\left(\frac{1}{(k+\frac{3}{2})\pi}\right) = -1$, $f\left(\frac{1}{(k+\frac{1}{2})\pi}\right) = 1$. Therefore for any $a_1 > 0$, the upper Riemann sum and the lower Riemann sum will never be equal in $[0,a_1]$ (and there is a partition $P$ such that $U(P, f) = L(P, f)$ on $[a_1, 1]$ by continuity), hence $f$ is not Riemann integrable. $\qed$
        \item False. Notice that the outer measure on $B(\epsilon)$ is $\epsilon$, and $[0,1]$ has outer measure 1. By the fact that if $A \subseteq B$ then $m^*(A) \leq m^*(B)$, we have that for $\epsilon < 1$, $B$ will never cover $[0,1]$. $\qed$
        \item 
        \begin{enumerate}
            \item 
            \item We guess that the value is $f(1)$, as the function $K_n(x)$ became concentrate at 1 as $n \to \infty$. We proof our claim below. \\
            First notice that 
            \[
            \int_0^1 K_n(x) = 1   \tag{1} 
            \]
            and
            \[
            \int_0^{1-\delta} K_n(x) \to 0 \text{ as } n \to \infty \tag{2}
            \]
            for every $\delta < 1$. Now as $f$ is continuous, for any $\epsilon > 0$ we can find $\delta > 0$ such that
            \[
            |f(x)-f(1)| < \epsilon \text{ whenever } |1-x| < \delta
            \]
            therefore
            \begin{align*}
            \left|\int^1_0 K_n(x) f(x) dx - f(1) \right| &= \left|\int^1_0 K_n(x) f(x) - f(1) dx \right| \overset{(1)}{=} \left|\int^1_0 K_n(x) (f(x) -  f(1)) dx\right| \\
            &\leq \int^1_{1-\delta} |K_n(x) (f(x) -  f(1))| dx + \int^{1-\delta}_0 |K_n(x) (f(x) -  f(1))| dx \\
            &\leq \epsilon + \int^{1-\delta}_0 2M K_n(x) \overset{(2)}{\to} \epsilon \text{ as } n \to \infty
            \end{align*}
            where $M$ is the number such that $|f(x)| \leq M$ for all $x \in [0,1]$. By the arbitrariness of $\epsilon$, we conclude that $\int^1_0 K_n(x) f(x) dx = f(1). \qed$
        \end{enumerate}
        \item 
        $f$ is integrable means that for every $\epsilon > 0$, there exists a partition $P = \{a=x_0, x_1, \cdots x_n = b\}$ such that
        \[
        \left\vert \int_a^b f(x)dx - \sum_{i=1}^n m_i (x_i - x_{i-1}) \right\vert < \epsilon
        \]
        where $m_i$ is the minimum of $f$ on $[x_{i-1}, x_i]$. This is the same as 
        \[
        0 \leq \int_a^b f(x) - g(x) < \epsilon	
        \]
        where $g(x) = \sum_{i=1}^n m_i \chi_{[x_{i-1}, x_i]}.$ Then 
        \begin{align*}
        \int_a^b f(x) \sin nx \; dx &\leq \int_a^b (f(x) - g(x)) \sin nx\; dx + \int_a^b g(x) \sin nx \;dx \\
        &\leq \int_a^b (f(x) - g(x))\;dx + \left\vert \frac{1}{n} \sum_i m_i (\cos n x_i - \cos n x_{i-1} ) \right\vert
        \end{align*}
        the former part is less than $\epsilon$, and the latter can be arbitrary small (we can choose $\epsilon$) as we take $n$ large enough. So the sum is less than $2\epsilon$, and by the arbitrariness of $\epsilon$, we conclude that
        \[
        \lim_{n \to \infty} \int_a^b f(x) \sin nx \; dx = 0. \qed
        \]
        \item 
        \begin{enumerate}
            \item Arezla-Ascoil Theorem states that a sequence of functions $\{f_n\}_{n \in \mathbb{N}}$ is uniform bounded and equicontinuous if and only if $\{f_n\}_{n \in \mathbb{N}}$ admits a subsequence that converges uniformly.
            \item This problem is flawed, as we can find a counterexample
            \[
                f_n(x) = -n. \qed
            \]
        \end{enumerate}
        \item Consider a $C^\infty$ function $F$
        \[
        F(x,y,z) = x^3+y^3+z^3-3xyz-1.
        \]
        \begin{enumerate}
            \item We calculate the partial derivative
            \[
            \frac{\partial F}{\partial z}  = 3z^2 - 3xy	
            \]
            and since $\frac{\partial F}{\partial z} (0,0,1) = 3 \neq 0$, by Implict Function Theorem there exists a neighborhood near $(0,0,1)$ such that $z = z(x,y)$, and it has the same differentiabiliy as $F$, i.e. $C^\infty$.
            \item We calculate the partial derivative
            \[
            \frac{\partial F}{\partial x}  = 3x^2 - 3yz	
            \]
            and since $\frac{\partial F}{\partial z} (0,0,1) = 0$, $x$ cannot be represent as a function $x=x(y,z)$. 
        \end{enumerate}
        \item
        \begin{enumerate}
            \item 
            The left hand side can be directly calculated as
            \begin{align*}
                \lim_{h \to 0} \int_c^d \frac{f(x+h, y) - f(x,y) }{h} dy
            \end{align*}
            and if we can pass the limit through the integral, then the assertion is proved. By the continuity of $f$, we can use mean value theorem to find $z \in [x,x+h]$ such that 
            \[
                f_h(z,y) = f_d(x,y) =  \frac{f(x+d, y) - f(x,y)}{d}
            \]
            where $f_d$ is the difference defined at the right. As $\frac{\partial f}{\partial x}$ is continuous and hence bounded in $[a,b]$, we have that for every sequence $\{n_k\} \to 0$, the sequence $\{f_{n_k}(x,y)\}$ is \textit{uniformly} bounded (as it does not depend on $n_k$ by the equation above) and converges pointwise to $\frac{\partial f}{\partial x}$ (by the existence of the partial derivative), therefore by Bounded convergence theorem, the limit can be passed through the integral sign, which proved the assertion.
    
            All the needed additional requirement in the proof above are the continuity of $f$ and $\dfrac{\partial f}{\partial x}$.
            \item The equation can be rewritten as 
            \[
            \lim_{d \to 0} \lim_{n \to \infty} \int_0^n \frac{f(x+d, y) - f(x,y)}{d} dy = \lim_{n \to \infty} \int_0^n \lim_{d \to 0} \frac{f(x+d, y) - f(x,y)}{d} dy
            \]
        \end{enumerate}
    \end{enumerate}